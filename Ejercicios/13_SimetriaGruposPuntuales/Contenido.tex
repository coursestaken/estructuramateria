\renewcommand{\labelitemi}{$\bullet$}

Asigne a cada una de las siguientes moléculas su grupo puntual.

% ~ ~ ~ ~ ~ ~ ~ ~ ~ ~ ~

\subsection*{a. $\textbf{O = C = C = C = O}$ (linear)}

\[ \scalebox{1.7}{$\boxed{D_{\infty h}}$} \]

\begin{itemize}
    \item Es lineal.
    \item Tiene un centro de inversión.
\end{itemize}

\iffalse
Es lineal.
Tiene un centro de inversión.

D_inf_h
\fi

\begin{center}
    \rule{15cm}{0.4pt}
\end{center}

% ~ ~ ~ ~ ~ ~ ~ ~ ~ ~ ~

\subsection*{b. $\textbf{HF}$}

\[ \scalebox{1.7}{$\boxed{C_{\infty v}}$} \]

\begin{itemize}
    \item Es lineal.
    \item No tiene centro de inversión.
\end{itemize}

\iffalse
Es lineal.
No tiene centro de inversión.

C_inf_v
\fi

\begin{center}
    \rule{15cm}{0.4pt}
\end{center}

% ~ ~ ~ ~ ~ ~ ~ ~ ~ ~ ~

\subsection*{c. $\textbf{IF}_7$}

\[ \scalebox{1.7}{$\boxed{D_{5h}}$} \]

\begin{itemize}
    \item No es lineal.
    \item Tiene un $C_5$ solamente por lo tanto no tiene mas de dos $C_n$ con $n \geq 3$.
    \item Tiene un $C_5$.
    \item Si tiene ejes $C_2$ perpendiculares al eje principal.
    \item Si tiene un plano de reflexión $\sigma_h$.
\end{itemize}

\iffalse
No es lineal.
Tiene un $C_5$ solamente por lo tanto no tiene mas de dos $C_n$ con $n \geq 3$.
Tiene un $C_5$.
Si tiene ejes $C_2$ perpendiculares al eje principal.
Si tiene un plano de reflexión $\sigma_h$.

D_5h
\fi

\begin{center}
    \rule{15cm}{0.4pt}
\end{center}

% ~ ~ ~ ~ ~ ~ ~ ~ ~ ~ ~

\subsection*{d. $\textbf{XeO}_2 \textbf{F}_2$}

\[ \scalebox{1.7}{$\boxed{C_{2h}}$} \]

\begin{itemize}
    \item No es lineal.
    \item No tiene $C_n$ con $n \geq 3$.
    \item Tiene $C_2$.
    \item No tiene $C_2$ perpendiculares al eje principal.
    \item Tiene plano $\sigma_h$.
\end{itemize}

\iffalse
No es lineal.
No tiene $C_n$ con $n \geq 3$.
Tiene $C_2$.
No tiene $C_2$ perpendiculares al eje principal.
Tiene plano $\sigma_h$.

C_2h
\fi

\begin{center}
    \rule{15cm}{0.4pt}
\end{center}

% ~ ~ ~ ~ ~ ~ ~ ~ ~ ~ ~

\subsection*{e. $\textbf{TeCl}_4$}

\[ \scalebox{1.7}{$\boxed{C_{2h}}$} \]

\begin{itemize}
    \item No es lineal.
    \item No tiene $C_n$ con $n \geq 3$.
    \item Tiene $C_2$.
    \item No tiene $C_2$ perpendiculares al eje principal.
    \item Tiene plano $\sigma_h$.
\end{itemize}

\iffalse
No es lineal.
No tiene $C_n$ con $n \geq 3$.
Tiene $C_2$.
No tiene $C_2$ perpendiculares al eje principal.
Tiene plano $\sigma_h$.

C_2h
\fi

\begin{center}
    \rule{15cm}{0.4pt}
\end{center}

% ~ ~ ~ ~ ~ ~ ~ ~ ~ ~ ~

\subsection*{f.}

\[ \scalebox{1.7}{$\boxed{C_s}$} \]

\begin{itemize}
    \item No es lineal.
    \item No tiene $C_n$ con $n \geq 3$.
    \item No tiene $C_2$ perpendiculares al eje principal.
    \item Tiene un plano de reflexión que pasa por los tres átomos.
\end{itemize}

\iffalse
No es lineal.
No tiene $C_n$ con $n \geq 3$.
No tiene $C_2$ perpendiculares al eje principal.
Tiene un plano de reflexión que pasa por los tres átomos.

C_s
\fi

\begin{center}
    \rule{15cm}{0.4pt}
\end{center}

% ~ ~ ~ ~ ~ ~ ~ ~ ~ ~ ~

\subsection*{g. \textit{trans}-dicloroetileno}

\[ \scalebox{1.7}{$\boxed{C_{2h}}$} \]

\begin{itemize}
    \item No es lineal.
    \item No tiene ejes de rotación $C_n$ con $n \geq 3$.
    \item Tiene un $C_2$ sobre el eje que sale del plano.
    \item No tiene $C_2$ perpendiculares al eje principal.
    \item Tiene un plano de reflexion horizontal $\sigma_h$.
\end{itemize}

\iffalse
No es lineal.
No tiene ejes de rotación $C_n$ con $n \geq 3$.
Tiene un $C_2$ sobre el eje que sale del plano.
No tiene $C_2$ perpendiculares al eje principal.
Tiene un plano de reflexion horizontal $\sigma_h$.

C_2h
\fi

\begin{center}
    \rule{15cm}{0.4pt}
\end{center}

% ~ ~ ~ ~ ~ ~ ~ ~ ~ ~ ~

\subsection*{h. Ciclopropano}

\[ \scalebox{1.7}{$\boxed{D_{3h}}$} \]

\begin{itemize}
    \item No es lineal.
    \item No tiene mas de dos rotaciones $C_n$ con $n \geq 3$.
    \item Tiene un $C_3$.
    \item Si tiene rotaciones $C_2$ perpendiculares al eje principal.
    \item Tiene un plano de reflexión horizontal $\sigma_h$.
\end{itemize}

\iffalse
No es lineal.
No tiene mas de dos rotaciones $C_n$ con $n \geq 3$.
Tiene un $C_3$.
Si tiene rotaciones $C_2$ perpendiculares al eje principal.
Tiene un plano de reflexión horizontal $\sigma_h$.

D_3h
\fi

\begin{center}
    \rule{15cm}{0.4pt}
\end{center}

% ~ ~ ~ ~ ~ ~ ~ ~ ~ ~ ~

\subsection*{i. Ciclopropeno}

\[ \scalebox{1.7}{$\boxed{C_2}$} \]

\begin{itemize}
    \item No es lineal.
    \item No tiene mas de dos rotaciones $C_n$ con $n \geq 3$.
    \item Tiene un $C_2$.
    \item No tiene rotaciones $C_2$ perpendiculares al eje principal.
    \item No tiene plano de reflexión horizontal $\sigma h$.
    \item No tiene planos de reflexión verticales.
    \item No es $S_{2n}$ con $n=2$.
\end{itemize}

\iffalse
No es lineal.
No tiene mas de dos rotaciones $C_n$ con $n \geq 3$.
Tiene un $C_2$.
No tiene rotaciones $C_2$ perpendiculares al eje principal.
No tiene plano de reflexión horizontal $\sigma h$.
No tiene planos de reflexión verticales.
No es $S_{2n}$ con $n=2$.

C_2
\fi

\begin{center}
    \rule{15cm}{0.4pt}
\end{center}

% ~ ~ ~ ~ ~ ~ ~ ~ ~ ~ ~

\subsection*{j. Aziridina}

\[ \scalebox{1.7}{$\boxed{C_s}$} \]

\begin{itemize}
    \item No es lineal.
    \item No tiene mas de dos rotaciones $C_n$ con $n \geq 3$.
    \item No tiene $C_2$.
    \item Tiene un plano de reflexión vertical.
\end{itemize}

\iffalse
No es lineal.
No tiene mas de dos rotaciones $C_n$ con $n \geq 3$.
No tiene $C_2$.
Tiene un plano de reflexión vertical.

C_s
\fi

\begin{center}
    \rule{15cm}{0.4pt}
\end{center}

% ~ ~ ~ ~ ~ ~ ~ ~ ~ ~ ~

\subsection*{k. $\textbf{Cr}_2\left(\textbf{CO}\right)^{2-}_{10}$}

\[ \scalebox{1.7}{$\boxed{C_{4v}}$} \]

\begin{itemize}
    \item No es lineal.
    \item No tiene mas de dos rotaciones $C_n$ con $n \geq 3$.
    \item No tiene $C_2$ perpendiculares al eje principal.
    \item No tiene plano de reflexión horizontal.
    \item Tiene cuatro planos verticales de reflexión.
\end{itemize}

\iffalse
No es lineal.
No tiene mas de dos rotaciones $C_n$ con $n \geq 3$.
No tiene $C_2$ perpendiculares al eje principal.
No tiene plano de reflexión horizontal.
Tiene cuatro planos verticales de reflexión.

C_4v
\fi

\begin{center}
    \rule{15cm}{0.4pt}
\end{center}

% ~ ~ ~ ~ ~ ~ ~ ~ ~ ~ ~

\subsection*{l. $\textbf{HCr}_2\left(\textbf{CO}\right)^{-}_{10}$}

\[ \scalebox{1.7}{$\boxed{D_{4h}}$} \]

\begin{itemize}
    \item No es lineal.
    \item Solo tiene un $C_4$.
    \item Tiene dos $C_2$ perpendiculares al $C_4$.
    \item Tiene un plano de reflexión horizontal $\sigma_h$.
\end{itemize}

\iffalse
No es lineal.
Solo tiene un $C_4$.
Tiene dos $C_2$ perpendiculares al $C_4$.
Tiene un plano de reflexión horizontal $\sigma_h$.

D_4h
\fi

\begin{center}
    \rule{15cm}{0.4pt}
\end{center}

% ~ ~ ~ ~ ~ ~ ~ ~ ~ ~ ~

\subsection*{m. $\textbf{Pt}_2\textbf{Cl}^{2-}_{6}$}

\[ \scalebox{1.7}{$\boxed{D_{2h}}$} \]

\begin{itemize}
    \item No es lineal.
    \item No tiene rotaciones $C_n$ con $n \geq 3$.
    \item Tiene $C_2$ perpendiculares al eje principal.
    \item Tiene un plano de reflexión horizontal $\sigma_h$.
\end{itemize}

\iffalse
No es lineal.
No tiene rotaciones $C_n$ con $n \geq 3$.
Tiene $C_2$ perpendiculares al eje principal.
Tiene un plano de reflexión horizontal $\sigma_h$.

D_2h
\fi

\iffalse
\begin{center}
    \rule{15cm}{0.4pt}
\end{center}
\fi

% ~ ~ ~ ~ ~ ~ ~ ~ ~ ~ ~

\subsection*{n. Fósforo blanco, $\textbf{P}_4$}

\[ \scalebox{1.7}{$\boxed{T_d}$} \]

\begin{itemize}
    \item No es lineal.
    \item Tiene cuatro $C_3$.
    \item No tiene punto de inversión.
\end{itemize}

\iffalse
No es lineal.
Tiene cuatro $C_3$.
No tiene punto de inversión.

T_d
\fi

\begin{center}
    \rule{15cm}{0.4pt}
\end{center}

% ~ ~ ~ ~ ~ ~ ~ ~ ~ ~ ~

\subsection*{o. Cubano, $\textbf{C}_8\textbf{H}_8$}

\[ \scalebox{1.7}{$\boxed{O_h}$} \]

\begin{itemize}
    \item No es lineal.
    \item Tiene más de dos rotaciones $C_n$ con $n \geq 3$.
    \item Tiene centro de inversión.
    \item No tiene $C_5$.
\end{itemize}

\iffalse
No es lineal.
Tiene más de dos rotaciones $C_n$ con $n \geq 3$.
Tiene centro de inversión.
No tiene $C_5$.

O_h
\fi

\begin{center}
    \rule{15cm}{0.4pt}
\end{center}

% ~ ~ ~ ~ ~ ~ ~ ~ ~ ~ ~

\subsection*{p. Tetrafluorocubano}

\[ \scalebox{1.7}{$\boxed{D_2}$} \]

\begin{itemize}
    \item No es lineal.
    \item No tiene $C_n$ con $n \geq 3$.
    \item Tiene $C_2$.
    \item Tiene $C_2$ perpendiculares al principal.
    \item No tiene un plano de reflexión horizontal $\sigma_h$.
    \item No tiene planos de reflexión diedral.
\end{itemize}

\iffalse
No es lineal.
No tiene $C_n$ con $n \geq 3$.
Tiene $C_2$.
Tiene $C_2$ perpendiculares al principal.
No tiene un plano de reflexión horizontal $\sigma_h$.
No tiene planos de reflexión diedral.

D_2
\fi