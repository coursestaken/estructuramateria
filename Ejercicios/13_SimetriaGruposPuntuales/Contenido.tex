Asigne a cada una de las siguientes moléculas su grupo puntual.

\subsection*{a. $\textbf{O = C = C = C = O}$ (linear)}

\begin{itemize}
    \item
\end{itemize}

%$$\boxed{C_a}$$

+ Es lineal
+ Tiene un centro de inversión.


D_inf_h

% ~ ~ ~ ~ ~ ~ ~ ~ ~ ~ ~

\subsection*{b. $\textbf{HF}$}

+ Es lineal
+ No tiene centro de inversión.


C_inf_v

% ~ ~ ~ ~ ~ ~ ~ ~ ~ ~ ~

\subsection*{c. $\textbf{IF}_7$}

+ No es lineal
+ Tiene C_5 solamente por lo tanto no tiene mas de dos C_n con n>3
+ Tiene un C_5
+ Si tiene ejes C_2 perpendiculares al eje principal
+ Si tiene un plano de reflexion \sigma_h


D_5h

% ~ ~ ~ ~ ~ ~ ~ ~ ~ ~ ~

\subsection*{d. $\textbf{XeO}_2 \textbf{F}_2$}

+ No es lineal
+ No tiene C_n con n>3
+ Tiene C_2
+ No tiene C_2 perpendiculares al eje principal
+ Tiene plano \sigma_h


C_2h

% ~ ~ ~ ~ ~ ~ ~ ~ ~ ~ ~

\subsection*{e. $\textbf{TeCl}_4$}

+ No es lineal
+ No tiene C_n con n>3
+ Tiene C_2
+ No tiene C_2 perpendiculares al eje principal
+ Tiene plano \sigma_h


C_2h

% ~ ~ ~ ~ ~ ~ ~ ~ ~ ~ ~

\subsection*{f.}

+ No es lineal
+ No tiene ejes de rotacion C_n con n>=3
+ No tiene ejes de rotación
+ Tiene un plano de reflexión que pasa por los tres átomos.


C_s

% ~ ~ ~ ~ ~ ~ ~ ~ ~ ~ ~

\subsection*{g. }

+ No es lineal
+ No tiene ejes de rotacion C_n con n>=3
+ Tiene un C_2, con el eje que sale del plano
+ No tiene ejes C_2 perpendiculares al eje principal
+ Tiene un plano de reflexion horizontal \sigma_h


C_2h


% ~ ~ ~ ~ ~ ~ ~ ~ ~ ~ ~

\subsection*{h. }

+ No es lineal
+ No tiene mas de dos ejes de rotacion C_n con n>=3
+ Tiene un C_3
+ Si tiene ejes de rotacion C_2 perpendiculares al eje principal
+ Tiene un plano de reflexión horizontal

D_3h

% ~ ~ ~ ~ ~ ~ ~ ~ ~ ~ ~

\subsection*{i. }

+ No es lineal
+ No tiene mas de dos ejes de rotacion C_n con n>=3
+ Tiene un C_2
+ No tiene ejes de rotacion C_2 perpendiculares al eje principal
+ No tiene plano de reflexion horizontal
+ No tiene planos de reflexion verticales
+ Si un enlace es hacia arriba y otro hacia abajo no tiene S2n


C_2


% ~ ~ ~ ~ ~ ~ ~ ~ ~ ~ ~

\subsection*{j. }

No es lineal
+ No tiene mas de dos ejes C_n con n>3
+ No tiene C_2
+ Tiene un plazo de reflexion verticales

C_s

% ~ ~ ~ ~ ~ ~ ~ ~ ~ ~ ~

\subsection*{k. }

No es lineal
Tiene un C_4 a lo largo del eje longitudinal no mas de dos
No tiene C_2 perpendiculares al eje principal
No tiene plano de reflexion horizontal
Tiene cuatroplanos de reflexion verticales

C_4v

% ~ ~ ~ ~ ~ ~ ~ ~ ~ ~ ~

\subsection*{l. }

No es lineal
Solo tiene un C4
Tiene 2 C_2 perpendiculares a C_4
Tiene un plano de reflexion horizontal

D_4h

% ~ ~ ~ ~ ~ ~ ~ ~ ~ ~ ~

\subsection*{m. }

No es lineal
No tiene C_n con n>=3
Tiene C2
Tiene Sigma h

D_2h

% ~ ~ ~ ~ ~ ~ ~ ~ ~ ~ ~

\subsection*{n. }

No s lineal
Tiene 4 C_3
No tiene punto de inversion


T_d

% ~ ~ ~ ~ ~ ~ ~ ~ ~ ~ ~

\subsection*{o. }

No es lineal
Tiene mas de dos C_4
Tiene centro de inversion 
No tiene C5

O_h

% ~ ~ ~ ~ ~ ~ ~ ~ ~ ~ ~

\subsection*{p. }

No es lineal
No tiene C_n con n>=3
Tiene C_2
Tiene C_2 perpendiculares al principal
No tiene Sigma H
No tiene planos de reflexion diedral

D_2